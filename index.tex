% Options for packages loaded elsewhere
\PassOptionsToPackage{unicode}{hyperref}
\PassOptionsToPackage{hyphens}{url}
\PassOptionsToPackage{dvipsnames,svgnames,x11names}{xcolor}
%
\documentclass[
  letterpaper,
  DIV=11,
  numbers=noendperiod]{scrartcl}

\usepackage{amsmath,amssymb}
\usepackage{iftex}
\ifPDFTeX
  \usepackage[T1]{fontenc}
  \usepackage[utf8]{inputenc}
  \usepackage{textcomp} % provide euro and other symbols
\else % if luatex or xetex
  \usepackage{unicode-math}
  \defaultfontfeatures{Scale=MatchLowercase}
  \defaultfontfeatures[\rmfamily]{Ligatures=TeX,Scale=1}
\fi
\usepackage{lmodern}
\ifPDFTeX\else  
    % xetex/luatex font selection
\fi
% Use upquote if available, for straight quotes in verbatim environments
\IfFileExists{upquote.sty}{\usepackage{upquote}}{}
\IfFileExists{microtype.sty}{% use microtype if available
  \usepackage[]{microtype}
  \UseMicrotypeSet[protrusion]{basicmath} % disable protrusion for tt fonts
}{}
\makeatletter
\@ifundefined{KOMAClassName}{% if non-KOMA class
  \IfFileExists{parskip.sty}{%
    \usepackage{parskip}
  }{% else
    \setlength{\parindent}{0pt}
    \setlength{\parskip}{6pt plus 2pt minus 1pt}}
}{% if KOMA class
  \KOMAoptions{parskip=half}}
\makeatother
\usepackage{xcolor}
\setlength{\emergencystretch}{3em} % prevent overfull lines
\setcounter{secnumdepth}{-\maxdimen} % remove section numbering
% Make \paragraph and \subparagraph free-standing
\makeatletter
\ifx\paragraph\undefined\else
  \let\oldparagraph\paragraph
  \renewcommand{\paragraph}{
    \@ifstar
      \xxxParagraphStar
      \xxxParagraphNoStar
  }
  \newcommand{\xxxParagraphStar}[1]{\oldparagraph*{#1}\mbox{}}
  \newcommand{\xxxParagraphNoStar}[1]{\oldparagraph{#1}\mbox{}}
\fi
\ifx\subparagraph\undefined\else
  \let\oldsubparagraph\subparagraph
  \renewcommand{\subparagraph}{
    \@ifstar
      \xxxSubParagraphStar
      \xxxSubParagraphNoStar
  }
  \newcommand{\xxxSubParagraphStar}[1]{\oldsubparagraph*{#1}\mbox{}}
  \newcommand{\xxxSubParagraphNoStar}[1]{\oldsubparagraph{#1}\mbox{}}
\fi
\makeatother


\providecommand{\tightlist}{%
  \setlength{\itemsep}{0pt}\setlength{\parskip}{0pt}}\usepackage{longtable,booktabs,array}
\usepackage{calc} % for calculating minipage widths
% Correct order of tables after \paragraph or \subparagraph
\usepackage{etoolbox}
\makeatletter
\patchcmd\longtable{\par}{\if@noskipsec\mbox{}\fi\par}{}{}
\makeatother
% Allow footnotes in longtable head/foot
\IfFileExists{footnotehyper.sty}{\usepackage{footnotehyper}}{\usepackage{footnote}}
\makesavenoteenv{longtable}
\usepackage{graphicx}
\makeatletter
\def\maxwidth{\ifdim\Gin@nat@width>\linewidth\linewidth\else\Gin@nat@width\fi}
\def\maxheight{\ifdim\Gin@nat@height>\textheight\textheight\else\Gin@nat@height\fi}
\makeatother
% Scale images if necessary, so that they will not overflow the page
% margins by default, and it is still possible to overwrite the defaults
% using explicit options in \includegraphics[width, height, ...]{}
\setkeys{Gin}{width=\maxwidth,height=\maxheight,keepaspectratio}
% Set default figure placement to htbp
\makeatletter
\def\fps@figure{htbp}
\makeatother

\KOMAoption{captions}{tableheading}
\makeatletter
\@ifpackageloaded{caption}{}{\usepackage{caption}}
\AtBeginDocument{%
\ifdefined\contentsname
  \renewcommand*\contentsname{Table des matières}
\else
  \newcommand\contentsname{Table des matières}
\fi
\ifdefined\listfigurename
  \renewcommand*\listfigurename{Liste des Figures}
\else
  \newcommand\listfigurename{Liste des Figures}
\fi
\ifdefined\listtablename
  \renewcommand*\listtablename{Liste des Tables}
\else
  \newcommand\listtablename{Liste des Tables}
\fi
\ifdefined\figurename
  \renewcommand*\figurename{Figure}
\else
  \newcommand\figurename{Figure}
\fi
\ifdefined\tablename
  \renewcommand*\tablename{Table}
\else
  \newcommand\tablename{Table}
\fi
}
\@ifpackageloaded{float}{}{\usepackage{float}}
\floatstyle{ruled}
\@ifundefined{c@chapter}{\newfloat{codelisting}{h}{lop}}{\newfloat{codelisting}{h}{lop}[chapter]}
\floatname{codelisting}{Listing}
\newcommand*\listoflistings{\listof{codelisting}{Liste des Listings}}
\makeatother
\makeatletter
\makeatother
\makeatletter
\@ifpackageloaded{caption}{}{\usepackage{caption}}
\@ifpackageloaded{subcaption}{}{\usepackage{subcaption}}
\makeatother

\ifLuaTeX
\usepackage[bidi=basic]{babel}
\else
\usepackage[bidi=default]{babel}
\fi
\babelprovide[main,import]{french}
% get rid of language-specific shorthands (see #6817):
\let\LanguageShortHands\languageshorthands
\def\languageshorthands#1{}
\ifLuaTeX
  \usepackage{selnolig}  % disable illegal ligatures
\fi
\usepackage{bookmark}

\IfFileExists{xurl.sty}{\usepackage{xurl}}{} % add URL line breaks if available
\urlstyle{same} % disable monospaced font for URLs
\hypersetup{
  pdftitle={Présentation},
  pdflang={fr},
  colorlinks=true,
  linkcolor={blue},
  filecolor={Maroon},
  citecolor={Blue},
  urlcolor={Blue},
  pdfcreator={LaTeX via pandoc}}


\title{Présentation}
\author{}
\date{}

\begin{document}
\maketitle


\paragraph{Les auteurs}\label{les-auteurs}

\begin{itemize}
\tightlist
\item
  \href{https://perso.univ-rennes2.fr/pierre-andre.cornillon}{Pierre-André
  Cornillon}
\item
  Nicolas Hengartner
\item
  \href{https://www.researchgate.net/profile/E-Matzner-Lober}{Éric
  Matzner-Løber}
\item
  \href{https://rouviere.pages.math.cnrs.fr}{Laurent Rouvière}
\end{itemize}

\paragraph{Descriptif}\label{descriptif}

\begin{itemize}
\tightlist
\item
  \href{quatriemeV3.pdf}{4e de couverture}
\item
  \href{avant-propos_V3.pdf}{Avant-propos}
\item
  \href{sommaire_v3.pdf}{Sommaire détaillé}
\end{itemize}

\paragraph{Boutique en ligne}\label{boutique-en-ligne}

\href{https://laboutique.edpsciences.fr/produit/1335/9782759831463/regression-avec-r}{Par
ici}

\section{Résumé}\label{ruxe9sumuxe9}

Cet ouvrage expose de manière détaillée, exemples à l'appui, différentes
façons de répondre à un des problèmes statistiques les plus courants :
la régression. Cette nouvelle édition se décompose en cinq parties.

\begin{enumerate}
\def\labelenumi{\arabic{enumi}.}
\item
  La première donne les grands principes des régressions simple et
  multiple par moindres carrés. Les fondamentaux de la méthode, tant au
  niveau des choix opérés que des hypothèses et leur utilité, sont
  expliqués.
\item
  La deuxième partie est consacrée à l'inférence et présente les outils
  permettant de vérifier les hypothèses mises en œuvre. Les techniques
  d'analyse de la variance et de la covariance sont également présentées
  dans cette partie.
\item
  Le cas de la grande dimension est ensuite abordé dans la troisième
  partie. Différentes méthodes de réduction de la dimension telles que
  la sélection de variables, les régressions sous contraintes (lasso,
  elasticnet ou ridge) et sur composantes (PLS ou PCR) sont notamment
  proposées. Un dernier chapitre propose des algorithmes, basés sur des
  méthodes de rééchantillonnage comme l'apprentissage/validation ou la
  validation croisée, qui permettent d'établir une comparaison entre
  toutes ces méthodes.
\item
  La quatrième partie se concentre sur les modèles linéaires généralisés
  et plus particulièrement sur les régressions logistique et de Poisson
  avec ou sans technique de régularisation. Une section particulière est
  consacrée aux comparaisons de méthodes en classification supervisée.
  Elle introduit notamment des critères de performance pour scorer des
  individus comme les courbes ROC et lift et propose des stratégies de
  choix seuil (Younden, macro F1\ldots) pour les classer. Ces notions
  sont ensuite mises en œuvre sur des données réelles afin de
  sélectionner une méthode de prévision parmi plusieurs algorithmes
  basés sur des modèles logistiques (régularisés ou non). Une dernière
  section aborde le problème des données déséquilibrées qui est souvent
  rencontré en régression binaire.
\item
  Enfin, la dernière partie présente l'approche non paramétrique à
  travers les splines, les estimateurs à noyau et des plus proches
  voisins. La présentation témoigne d'un réel souci pédagogique des
  auteurs qui bénéficient d'une expérience d'enseignement auprès de
  publics très variés. Les résultats exposés sont replacés dans la
  perspective de leur utilité pratique grâce à l'analyse d'exemples
  concrets. Les commandes permettant le traitement des exemples sous R
  figurent dans le corps du texte.
\end{enumerate}

Enfin, chaque chapitre est complété par une suite d'exercices corrigés.




\end{document}
